\documentclass[12pt]{article}
\usepackage{amsfonts, epsfig}
\usepackage{amsmath}
\usepackage{wrapfig} % for wrapping text around figures and tables
\usepackage{graphicx}
\usepackage{fancyhdr}
\pagestyle{fancy}
\lfoot{\texttt{emamtm0067.github.io} / \texttt{ematm0044.github.io}}
\lhead{Question 1 - Visualization and analysis of the Palmer penguin dataset}
\rhead{\thepage}
\cfoot{}
\begin{document}

\subsection*{Q1 - Visualization and analysis of the Palmer dataset}

\begin{wraptable}{r}{0.7\textwidth} % {alignment}{width}
  \small
  \begin{center}
  \vspace{-2\baselineskip} % Remove space before the table
  \setlength{\abovecaptionskip}{5pt}
  \setlength{\belowcaptionskip}{5pt}
  \begin{tabular}{l|l|l}
  Attribute&Type&Values in the dataset\\
  \hline
  species&categorial&Adelie, Chinstrap, Gentoo\\
  island&categorial&Torgersen, Biscoe, Dream\\
  bill length&numerical&32.1mm - 59.6mm\\
  bill depth&numerical&13.1mm - 21.5mm\\
  flipper length&numerical&172mm - 231mm\\
  body mass&numerical&2700g - 6300g\\
  sex&categorial&Male, Female
  \end{tabular}
  \vspace{-1.5\baselineskip} % Remove space after the table
  \end{center} 
  \caption{Attributes of the Palmer penguin dataset}
  \vspace{-1\baselineskip} % Remove space after the table
  \label{tab:example}
\end{wraptable} 

The Palmer penguin dataset consists of 344 records of the physical attributes of three species of penguin living on three islands in Antarctica (Table~\ref{tab:example}) [1]. 

In this report, consideration is given to data cleaning and preparation, the dataset is explored through visualization and analysis is carried out to compare the accuracy performances of a small number of AI approaches. 

There is a table at Table~\ref{tab:example} and a figure at Fig.~\ref{fig:example}.




Here is an example of an equation:
\begin{equation}
  \pi=4\left(1-\frac{1}{3}+\frac{1}{5}-\frac{1}{7}\ldots\right)
\end{equation}
or
\begin{equation}
  \pi=4\sum_{n=0}^\infty\frac{(-1)^{n}}{2n+1}
\end{equation}


where $\pi$ can be written in line by using \$'s. Here is a vector:
\begin{equation}
\mathbf{x}=\left(\begin{array}{c}x_1\\x_2\end{array}\right)
\end{equation}
You can write in \textbf{bold}, or \textsl{italics} or \texttt{true
  type}, often the latter is used for specific commands or libraries in a
programming language, as in `I used \texttt{numpy} v1.23.4 to\ldots'. Notice the use of the left quote symbol found in the top left of the keyboard to get the left quote. There is also blackboard bold often used for things like $\mathbb{R}$ for real numbers and there is calligraphic for fancy things like $\mathcal{L}$ but this is becoming increasing irrelevant to what you are likely to need! 







\end{document}
